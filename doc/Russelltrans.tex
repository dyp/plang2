\documentclass[10pt,a4paper]{article}
\usepackage[utf8x]{inputenc}
\usepackage[russian]{babel}
\usepackage{ucs}
\usepackage{amsmath}
\usepackage{amsfonts}
\usepackage{amssymb}
\usepackage{makeidx}
\usepackage{verbatim}
\usepackage{listings}
\lstset{extendedchars=\true,inputencoding=utf8x}
\usepackage{indentfirst}
\addtolength{\parskip}{\baselineskip}

\lstdefinelanguage{P}{
morekeywords={var, int, nat, formula, module, if, else, lemma, exists, pre, post, measure},
escapechar=@
}
\lstset{language=P}
\begin{document}
\begin{center}
\textbf{
\begin{LARGE}
Автоматическое доказательство формул корректности
 предикатной программы в системе Russell.
\end{LARGE}
}
\end{center}
Для предикатной программы на внутреннем представлении строится набор теорий с формулами корректности предикатной программы. Во внутреннем представлении теория кодируется конструкцией МОДУЛЬ.

Задача проекта – обеспечить автоматическое доказательство лемм в теории, сгенерированной транслятором. Для этого необходимо реализовать трансляцию теории с внутреннего языка P на язык системы автоматического доказательства Russell. Далее леммы теории должны быть автоматически доказаны в системе Russell.

\section{ Теории модуля}

Программа составлена из модулей. Модуль является единицей компиляции и содержит набор глобальных описаний и определения предикатов.Глобальные объекты модуля (типы, константы, формулы, константы), используетмые в генерируемых теориях формируют главную теорию.

Для каждого определения предиката строится теория, имя которй совпадает с именем предиката.

Набор теорий, создаваемый генератором формул корректности, состоит из главной теории и теорий, сгенерированных для определений предикатов. Набор теорий должен быть замкнутым: для всякого вхождения имени (формулы, структурного типа, переменной, …) имеется соответствующее описание, за исключением имен примитивных типов. Набор теорий (подмодулей) доступен из специального поля конструкции МОДУЛЬ, полученной трансляцией исходной программы во внутреннее представление.

Теория во внутреннем представлении (конструкция МОДУЛЬ) состоит из формул и утверждений. Локальные объекты (типы, константы, переменные), используемые в формулах и утвержлениях, должны быть представлены как объекты генерируемого модуля. Глобальные объекты должны быть определены в других теориях, и эти теории должны присутствовать в списке импорта модуля. Если локальный объект строится на базе другого локального объекта, то последний должен быть определен ранее.

\section{Трансляция в Russell на примерах.}


Остановимся пока на конструкциях,используемых в примерах.

Описание структуры образа произвольной конструкции языка P в соответствующей конструкции языка спецификаций Russell реализуется в следующей форме:
\begin{flushleft}
\textbf{tr(<языковая конструкция>)  =  <образ констукции>}
\end{flushleft}

Логическая семантика оператора присваивания определяется очевидным образом:

\begin{flushleft}
\textbf{L(<переменная> = <выражение>)  =  tr(<переменная>) = tr(<выражение>)}
\end{flushleft}

Леммы в языке Russell представляются в виде проблем(теорем,требуюших автоматического доказательства):

\begin{flushleft}
\textbf{tr( lemma СПИСОК-ПОСЫЛОК => СПИСОК-ЗАКЛЮЧЕНИЙ )  = problem ИМЯ-ПРОБЛЕМЫ ( СПИСОК-АРГУМЕНТОВ ) \{ СПИСОК-ПОСЫЛОК bar СПИСОК-ЗАКЛЮЧЕНИЙ \} }

Формулы, используемых в языке P в качестве предусловий и постусловий необходимо определять в Russell следующим образом:
\begin{lstlisting}
constant
{
  symbol P ;
  ascii P ;
  latex P ;
}

rule rl-P ( var a : class , var b : class)
{
  term : wff = # P ( a , b ) ;
}

definition df-P (var a : class , var b : class )
{
  defiendum : wff = # P ( a , b ) ;
  definiens : wff = # ( ( ( a ≥ 0 ) @$\wedge$@ ( b ≥ 0 ) ) @$\wedge$@ ( ( a @ $∈ \Bbb{N}_{0} $ @ ) @$\wedge$@ ( b @$∈ \Bbb{N}_{0} $ @ ) ) ) ;
  -----------------
  prop : wff = |- ( defiendum ↔ definiens ) ;
}

\end{lstlisting}


\begin{flushleft}
Нам также понадобится определить операцию >  и  $\geqslant$:
\begin{lstlisting}
constant
{
  symbol > ;
  ascii > ;
  latex > ;
}

rule rlmore()
{
  term : class = #  > ;
}

constant
{
  symbol @$\geqslant$@ ;
  ascii @$\geqslant$@ ;
  latex @$\geqslant$@ ;
}

rule rlmoreq()
{
  term : class = #  @$\geqslant$@ ;
}
definition dfmore ()
{
  defiendum : class = # > ;
  definiens : class = # ( ( @$\Bbb{R} \times \Bbb{R}$@ ) ∖ @$\leqslant$ @) ;
  -----------------
  prop : wff = |- ( defiendum = definiens ) ;
}

definition dfmoreq ()
{
  defiendum : class = @$\geqslant$@ ;
  definiens : class = # ( ( @$\mathbb{R} \times \Bbb{R}$@ ) ∖ < ) ;
  -----------------
  prop : wff = |- ( defiendum = definiens ) ;
}
\end{lstlisting}
\end{flushleft}
\begin{flushleft}
Образами операций и отношений будут:
\end{flushleft}
\textbf{tr( <= )  =  $\leq $}

\textbf{tr( < )  = < }

\textbf{tr( not )  = $ \neg $}

\textbf{tr( \& )  = $ \wedge $}

\textbf{tr( - )  = - }
\end{flushleft}

\begin{flushleft}
Конструкции \textbf{$\neg a = b$ } в языке Russell существует  соответсвующая $a \neq b$.
\end{flushleft}
\begin{flushleft}
Нам также понадобится определить в Russell тип натуральных чисел(\textbf{nat}). Явно определенного типа натуральных чисел в Russell нет, но есть определенные в библиотеке множества $\Bbb{N}$ ( натуральные числа без нуля) и  $\Bbb{N}_{0}$ (натуральные числа с нулем). Отсюда:

\textbf{tr( nat A )  =  tr ( A ) $\in \Bbb{N}_{0}$}
\end{flushleft}

\begin{flushleft}
Рассмотрим первый пример. Программа умножения через сложение на языке предикатного программирования:
\end{flushleft}

\begin{flushleft}
\begin{lstlisting}
Умн( nat a, b: nat c)
 pre a ≥ 0 & b ≥ 0
  { if (a = 0) c = 0 else c = b + Умн(a - 1, b) }
 post} c = a * b measure a;
\end{lstlisting}
\begin{flushleft}
Теория после семантического анализа:
\end{flushleft}
\begin{lstlisting}
theory
  formula P(nat a, b) = a >= 0 & b >= 0;
  lemma P(a, b) & not a = 0 => a – 1 >= 0;
end
\end{lstlisting}
\end{flushleft}


\begin{flushleft}
Оттранслировав на язык Russell получим:
\end{flushleft}
\begin{flushleft}
\begin{lstlisting}
import math;
contents of math {

problem ex1 (var а : class, var b : class)
{
 hyp 1 : wff = |- ( P ( a , b ) @$\wedge$@ ( a @$≠$@ 0 ) ) ;
 -----------------
 prop : wff = |- ( ( a - 1 ) @$\geqslant$@ 0 )  ;
}
proof
 {
  step 1 : wff = ? |- ( ( a - 1 ) @$\geqslant$@ 0 ) ;
  qed prop = step 1 ;
 }
}
\end{lstlisting}
\end{flushleft}

\begin{flushleft}


Рассмотрим второй пример : \textbf{Наибольший общий делитель.}

Определим программу D(a, b: c) вычисления наибольшего общего делителя (НОД) положительных a и b. Свойство «x является делителем a» представляется следующим предикатом:
\end{flushleft}
\begin{center}
x – делитель a $\equiv divisor(x, a) \equiv \exists$ z ≥ 0. x * z = a .
\end{center}
\begin{flushleft}
Предикат divisor2(с, a, b) $\equiv$ divisor(с, a) \& divisor(с, b) определяет \textbf{с} в качестве делителя значений a и b. Свойство наибольшего общего делителя определяется предикатом:
\end{flushleft}
\begin{center}
НОД(c, a, b) $\equiv divisor2(c , a, b)  \&  \forall  x. (divisor2(x, a, b) \Rightarrow$ x ≤ c) .
\end{center}
Приведенная далее программа вычисления наибольшего общего делителя базируется на следующих известных свойствах НОД:
\begin{center}
a = b  $\Rightarrow$ НОД(a, a, b)
a > b \& НОД(c, a, b) $\Rightarrow$  НОД(c, a - b, b)
НОД(c, a, b) $\equiv$ НОД(c, b, a)
\end{center}
\begin{flushleft}
\begin{lstlisting}
D(nat a, b: nat c) @$\equiv$@ pre a ≥ 1 & b ≥ 1
{
 if (a = b) c = a
  else if (a < b) D(a, b-a: c)
  else D(a-b, b: c)
} post НОД(c, a, b) measure a + b;


@Теория после семантического анализа:@


theory
  formula P(nat a, b) = a >= 1 & b >= 1;
  lemma P(a, b) & a < b => b - a >= 0;
  lemma P(a, b) & not a < b => a - b >= 0;
end
\end{lstlisting}
\end{flushleft}
\begin{flushleft}
Оттранслировав на язык Russell получим:
\end{flushleft}

\begin{flushleft}
\begin{lstlisting}
import math;
contents of math
{
constant
{
 symbol P ;
 ascii P ;
 latex P ;
 }

rule rl-P (var a : class , var b : class)
{
  term : wff = # P ( a , b );
}

definition df-P (var a : class , var b : class )
{
  defiendum : wff = # P ( a , b ) ;
  definiens : wff = # ( ( ( a @$\geqslant$@ 1 ) @$\wedge$@ ( b @$\geqslant$@ 1 ) ) @$\wedge$@ ( ( a @$∈ \Bbb{N}_{0} $@ ) @$\wedge$@ ( b @$∈ \Bbb{N}_{0} $@ ) ) ) ;
  -----------------
  prop : wff = |- ( defiendum @$\leftrightarrow$@ definiens ) ;
}

problem nodexam1 (var a : class, var b : class)
{
 hyp 1 : wff = |- ( P ( a , b ) @$\wedge$@ ( a < b ) ) ;
 -----------------
 prop : wff = |- ( ( b - a ) @$\geq$@ 0) ;
}
proof
{
 step 1 : wff = ? |- ( ( b - a ) @$\geqslant$@ 0 ) ;
 qed prop = step 1 ;
}

problem nodexam2 (var a : class, var b : class)
{
  hyp 1 : wff = |- ( P ( a , b ) @$\wedge$@ @$\neg$@( a < b ) );
  -----------------
  prop : wff = |- ( ( a - b ) @$\geqslant$@ 0 ) ;
}
proof
 {
step 1 : wff = ? |- ( ( a - b ) @$\geqslant$@ 0 ) ;
qed prop = step 1 ;
 }
}
\end{lstlisting}
\end{flushleft}

\begin{flushleft}
Полученные в примерах проблемы должны быть автоматически доказаны в системе Russell.
\end{flushleft}

\section{Выражения}

\begin{flushleft}
Для условного выражения
\\
tr( с ? a : b )  =  \rm{if}( c , a , b )
\end{flushleft}
\begin{flushleft}
Для унарной op a и бинарной a op b операции, где op - операция, а a и b  -  выражения, реализуется отображения:
\end{flushleft}


\begin{flushleft}
tr( op a )  =  tr(op) (tr(a))

tr( a op b )  =  (tr(a)) tr(op) (tr(b))
\end{flushleft}

Отображения унарных операций следующие:
tr( + )  - унарный плюс отсутствует

tr( - )  = -𝐮

tr( ! )  =  $\neg$

\begin{flushleft}
Отображения бинарных операций следующие:

tr( \textbf{\^} )  =  $ \uparrow $

tr( * )  = $ \bullet $

tr( / )  = /

tr( a \% b )  = a $\rm{mod}$ b

tr( + )  =  +

tr( < )  =  <

tr( <= )  = $\leqslant$

tr( != )  =  $\neq $

tr( \& )  =  $\wedge$ для логических операндов

tr( a or b )  =  a $\vee$ b для логических

tr( => )  =  $\rightarrow$

tr( <=> )  =  $\leftrightarrow$

tr( a xor b )  =  a XOR b для логических.

Определим операцию XOR,так как в библиотеке Russell ее нет.
\begin{flushleft}
\begin{lstlisting}
constant
{
 symbol XOR ;
 ascii XOR ;
 latex XOR ;
}

rule rl-hor (var ph : wff , var ps : wff )
{
 term : wff = # ( ph XOR ps ) ;
}

definition df-xor (var ph : wff , var ps : wff )
{
 defiendum : wff = # ( ph XOR ps ) ;
 definiens : wff = # ( ( ph @$\vee$@ ps ) @$\wedge$@ @$\neg$@ ( ph @$\wedge$@ ps ) ) ;
 -----------------
 prop : wff = |- ( defiendum ↔ definiens ) ;
}
\end{lstlisting}
\end{flushleft}
\end{flushleft}
Для различных операций с множествами:

tr( <выражение1> + <выражение2> )  =  (tr(<выражение1>) $\bigcup$ tr(<выражение2>))

tr( <выражение1> or <выражение2> )  =  (tr(<выражение1>) $\bigcup$ tr(<выражение2>))

tr( <выражение1> - <выражение2> )  = (tr(<выражение1>) $\setminus $ tr(<выражение2>))

tr( <выражение1> \& <выражение2> )  =  tr(<выражение1>) $\bigcap$  tr(<выражение2>))

tr( <выражение1> in <выражение2> )  =  (tr(<выражение1>) $\in$ tr(<выражение2>))


\section{Типы}

tr(nat A)  =  tr(A) $\in \Bbb{N}_{0}$

tr(int A)  =  tr(A) $\in \Bbb{Z}$

tr(real A) =  tr(A)$\in \Bbb{R} $

Для определения типа bool нам понадобится ввести в библиотеку его описание:
\begin{lstlisting}
constant
{
  symbol @$\Bbb{B}$@;
  latex  mathbb\{B} ;
}

rule cn ()
{
 term : class = # @$\Bbb{B}$@ ;
}

definition tp-bool (var x : set)
{
  defiendum : class = # @$ \Bbb{B} $@;
  definiens : class = # { x | ( ( x = 1 ) @$\vee$@ ( x = 0 ) ) } ;
  -----------------
  prop : wff = |- ( defiendum = definiens ) ;
}
\end{lstlisting}

\par Для определения имени типа как параметра предиката или описания другого типа:
tr(type <имя типа>)  =  type tr(<имя типа>)

\section{Массивы}


Нужно задавать массив как сюрьекцию(отображение "на") из конечного подмножества натуральных чисел, содержащего все натуральные числа от единицы до числа, задающего размер массива, в множество, элементы которого имеют заданный тип.
(замечание: Индекс имеет произвольный конечный тип. Но мы можем пока в качестве демонстрационного примера рассматривать тип индекса как nat, так как с прочими типами будет аналогичная ситуация. Отдельный интерес представляет тип char. Дело в том, что в языке системы Russell для него невозможно корректно определить образ. Например, мы можем добавить в грамматику языка новую константу (назовем ее char) :

\begin{lstlisting}
constant
{
	symbol char ;
	ascii char ;
}

rule rl-Char ()
{
 term : class = # char ;
}

\end{lstlisting}
Переменная типа char может содержать в себе произвольный символ. Но где гарантия, что этот символ уже не задекларирован в библиотеке Russell( например символ $"$ в библиотеке определен как операция взятия образа функции ).
Положим, к примеру, что мы имеем описание массива:

\begin{center}
\textit{type Arr = array ( nat, 1..3 );}
\end{center}

В библиотеке языка Russell нужно определить объект, который бы задавал массив натуральных чисел размера 3. Определим его как сюрьекцию(отображение "на") из конечного подмножества натуральных чисел, содержащего все натуральные числа от единицы до трех, в множество, элементы которого имеют тип \textbf{nat}.
\begin{lstlisting}
constant
{
   symbol Array ;
   ascii Array ;
}

rule rl-Array (var A : set)
{
 term : set = # Array ( A , 3 ) ;
}

definition df-Array (var Ar : set, var A : set, var I : set)
{
 defiendum : set =  # Array ( A , 3 ) ;
 definiens : class =  # { Ar | ( ( Ar : I @$\twoheadrightarrow$@ A ) @$\wedge$@ ( I = ( 1 ... 3 ) ) @$\wedge$@ ( A @$\in \Bbb{N}_{0}$@ ) ) } ;
 -----------------
 prop : wff = |- ( defiendum = definiens ) ;
}
\end{lstlisting}
Теперь массив необходимо инициализировать:
    Arr ArrayTest = [ 3, 1, 4 ];
\begin{lstlisting}
constant
{
   symbol ArrayTest ;
   ascii ArrayTest ;
}

rule rl-Array ()
{
 term : class = # ArrayTest ;
}

definition df-ArrayNatTest (var A : set)
{
 defiendum : class =  # ArrayTest ;
 definiens : class =  # { Array ( A , 3 ) | ( ( ( Array ( A , 3 ) ` 2 ) = 1 )
 @$\wedge$@ ( ( Array ( A , 3 ) ` 1 ) = 3 ) @$\wedge$@ ( ( Array ( A , 3 ) ` 3 ) = 4 ) ) } ;
 -----------------
 prop : wff = |- ( defiendum = definiens ) ;
}
\end{lstlisting}
Но не будем забывать и про многомерные массивы. В этом случае будем задавать множество индексов как декартово произведение двух или более множеств( для двумерных и многомерных массивов соответственно ) :
\begin{lstlisting}
constant
{
   symbol ArrayM ;
   ascii ArrayM ;
}

rule rl-ArrayM (var A : set)
{
 term : set = # ArrayM ( A , 3 , 2 ) ;
}

definition df-ArrayM (var Ar : set, var A : set, var I : set, var X : set, var Y : set)
{
 defiendum : set =  # ArrayM ( A , 3 , 2 ) ;
 definiens : class =  # { Ar | ( ( Ar : I @$\twoheadrightarrow$@ A ) @$\wedge$@ ( ( X = ( 1 ... 3 ) )
 @$\wedge$@ ( Y = ( 1 ... 2 ) ) @$\wedge$@ ( I = ( X @$\times$@ Y ) ) ) @$\wedge$@ ( A @$\in \Bbb{N}_{0}$@ ) ) } ;
 -----------------
 prop : wff = |- ( defiendum = definiens ) ;
}
\end{lstlisting}

Инициализация идентична инициализации одномерных массивов.
\section{Структуры}

Пусть у нас есть объявление структуры:
\begin{center}

\textit{type Point = struct ( int x , y );}
\end{center}
Определим в Russelle структуру с двумя целыми полями:
\begin{lstlisting}
constant
{
   symbol Struct ;
   ascii Struct ;
}

rule rl-Struct (var x : set, var y : set)
{
 term : set = # Struct ( x , y ) ;
}

definition df-Struct (var x : set, var y : set)
{
 defiendum : set =  # Struct ( x , y ) ;
 definiens : class =  # { @$\langle$@ x , y @$\rangle$@ | ( ( x @$\in \Bbb{Z}$@ ) @$\wedge$@ ( y @$\in \Bbb{Z}$@ ) ) } ;
 -----------------
 prop : wff = |- ( defiendum = definiens ) ;
}

\end{lstlisting}
Инициализируем структуру:
\begin{center}
\textit{Point pt = (1,2);}
\end{center}
\begin{lstlisting}
constant
{
   symbol Pt ;
   ascii Pt ;
}

rule rl-StructTest ()
{
 term : class = # Pt ;
}

definition df-StructTest (var x : set, var y : set)
{
 defiendum : class =  # Pt ;
 definiens : class =  # { Struct ( x , y ) | ( ( x = 1 ) @$\wedge$@ ( y = 2 ) ) } ;
 -----------------
 prop : wff = |- ( defiendum = definiens ) ;
}
\end{lstlisting}
\end{document}
